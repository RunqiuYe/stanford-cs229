% HMC Math dept HW template example
% v0.04 by Eric J. Malm, 10 Mar 2005
\documentclass[12pt,letterpaper,boxed]{hmcpset}

% set 1-inch margins in the document
\usepackage[margin=1in]{geometry}
\usepackage{alltt}
\usepackage{amsfonts}
\usepackage{amsmath}
\usepackage{amssymb}
\usepackage{amsthm}
\usepackage{booktabs}
\usepackage{caption}
\usepackage{fancyhdr}
\usepackage{graphicx}
\usepackage{mathdots}
\usepackage{mathtools}
\usepackage{microtype}
\usepackage{multirow}
\usepackage{pdflscape}
\usepackage{pgfplots}
\usepackage{siunitx}
\usepackage{slashed}
\usepackage{tabularx}
\usepackage{tikz}
\usepackage{tkz-euclide}
\usepackage[normalem]{ulem}
\usepackage[all]{xy}
\usepackage{imakeidx}
\usepackage{enumerate}
\usepackage{physics}

% include this if you want to import graphics files with /includegraphics
\usepackage{graphicx}

\newcommand{\yy}{y^{(i)}}
\newcommand{\xx}{x^{(i)}}
\renewcommand{\tt}{t^{(i)}}

% info for header block in upper right hand corner
\name{Runqiu Ye}
\class{Stanford CS299}
\assignment{Problem Set \#1}
\duedate{06/23/2024}

\linespread{1.15}
\begin{document}

\problemlist{Problem Set \#1: Supervised Learning}

\begin{problem} [Problem 1]
    \textbf{Linear Classifiers (Logistic Regression and GDA)}
    
    Consider two datasets provided in the following files:
    \begin{enumerate}[i.]
    	\item \verb*|data/ds1_{train,valid},csv|
    	\item \verb*|data/ds2_{train,valid},csv|
    \end{enumerate}
	Each file contains $m$ examples, one example per row. The $i$-th row contains columns $\xx_0 \in \R$, $\xx_1 \in \R$ and $\yy \in \{0, 1\}$. Use logistic regression and GDA to perform binary classification. 
\end{problem}

\begin{solution}

\begin{enumerate}[(a)]
  \item Average empirical loss for logistic regression:
  \[
  J(\theta) = -\frac{1}{m} \sum_{i=1}^m \yy \log(h_\theta (\xx)) + (1-\yy)
    \log(1-h_\theta(\xx)),
  \]
  where $\yy \in \{0,1\}$, $h_\theta(\xx) = g(\theta^T x)$ and $g(z) = 1/(1+e^{-z})$.
  
  The gradient of the function
  \[
  \pdv{J}{\theta_j} = - \frac{1}{m} \sum_{i=1}^m (\yy - h_\theta(\xx)) \xx_j.
  \]
  It follows that
  \[
  \pdv{J}{\theta_k}{\theta_j} = \frac{1}{m} \sum_{i=1}^m h_\theta(\xx) (1-h_\theta(\xx)) \xx_k \xx_j.
  \]
  Hence, The Hessian $H$ of this function is
  \[
  H = \frac{1}{m} \sum_{i=1}^m  h_\theta(\xx) (1-h_\theta(\xx)) \xx (\xx)^T.
  \]
  
  Now, for any vector $z$, using Einstein's summation, we have
  \[
  \begin{aligned}
  z^T H z &= \frac{1}{m} \sum_{i=1}^m h_\theta(\xx) (1-h_\theta(\xx)) z_k \xx_k \xx_j z_j \\
  &= 
  \frac{1}{m} \sum_{i=1}^m h_\theta(\xx) (1-h_\theta(\xx)) (x^T z)^2\\
  & \geq 0
  \end{aligned}
  \]
  
  This shows that $H$ is PSD, and $J$ is convex.
  
  \item \textbf{Coding problem.}

  \item 
  To show that GDA results in a classifier that has a linear decision boundary, we want to show 
  \[
  p(y = 1 \mid x; \phi, \mu_0, \mu_1, \Sigma) = \frac{1}{1 + \exp(-(\theta^T x + \theta_0))}
  \]
  for some $\theta \in \R^n$ and $\theta_0 \in \R$ as functions of $\phi$, $\Sigma$, $\mu_0$, and $\mu_1$.
  We have
  \[
  \begin{aligned}
      p(y = 1 \mid x) &= \frac{p(x \mid y=1) p(y=1)}{p(x \mid y=1) p(y=1) + p(x\mid y = 0) p(y = 0)} \\
      &= \frac{\phi \exp(-\frac{1}{2} (x-\mu_1)^T \Sigma^{-1} (x-\mu_1) )}{\phi \exp(-\frac{1}{2} (x-\mu_1)^T \Sigma^{-1} (x-\mu_1)) + (1-\phi) \exp(-\frac{1}{2} (x-\mu_0)^T \Sigma^{-1} (x-\mu_0))} \\
      &= \frac{1}{1 + \frac{1-\phi}{\phi} \exp(-\frac{1}{2} (x-\mu_0)^T \Sigma^{-1} (x-\mu_0) + \frac{1}{2} (x-\mu_1)^T \Sigma^{-1} (x-\mu_1))} \\
      &= \frac{1}{1 +\frac{1-\phi}{\phi} \exp(-((\mu_1 - \mu_0)^T \Sigma^{-1} x + \frac{1}{2} (\mu_0^T \Sigma^{-1} \mu_0 - \mu_1^T \Sigma^{-1} \mu_1)))}.
  \end{aligned}
  \]
  This is the desired form, where
  \[
  \begin{aligned}
      \theta &= \Sigma^{-1} (\mu_1 - \mu_0), \\
      \theta_0 &= \frac{1}{2} (\mu_0^T \Sigma^{-1} \mu_0 - \mu_1^T \Sigma^{-1} \mu_1) - \log \frac{1-\phi}{\phi}.
  \end{aligned}
  \]
  
  \item The log-likelihood of the data is
  \[
  \begin{aligned}
      \ell(\phi, \mu_0, \mu_1, \Sigma) &= \log \prod_{i=1}^m p(\xx \mid \yy; \mu_0, \mu_1, \Sigma) p(\yy; \phi) \\
      &= \sum_{i=1}^m 1\{\yy=1\} \qty(-\frac{1}{2} (\xx-\mu_1)^T \Sigma^{-1} (\xx-\mu_1) + \log \phi) \\
      & \qquad + \sum_{i=1}^m 1\{\yy=0\} \qty(-\frac{1}{2} (\xx-\mu_0)^T \Sigma^{-1} (\xx-\mu_0) + \log (1-\phi))  \\
      & \qquad - \frac{m}{2} \log \abs{\Sigma} +  C,
  \end{aligned}
  \]
  where $C$ is some constant independent of the parameters.
  
  Let $\nabla_\phi \ell = 0$, we have
  \[
  \phi = \frac{1}{m} \sum_{i=1}^m 1\{\yy=1\}.
  \]
  Let $\nabla_{\mu_1} \ell = 0$, we have
  \[
  \sum_{i=1}^m 1\{\yy=1\} \Sigma^{-1}\xx = \sum_{i=1}^m 1\{\yy=1\} \Sigma^{-1} \mu_1,
  \]
  and thus
  \[
  \mu_1 = \frac{\sum_{i=1}^m 1\{\yy=1\} \xx}{\sum_{i=1}^m 1\{\yy=1\}}, \quad 
  \mu_0 = \frac{\sum_{i=1}^m 1\{\yy=0\} \xx}{\sum_{i=1}^m 1\{\yy=0\}}.
  \]
  To derive $\Sigma$, recall that $\nabla_{A} \log \abs{A} = (A^{-1})^T$, so we have
  \[
  \nabla_{\Sigma^{-1}} \ell = -\frac{m}{2} \Sigma^{-1} + \frac{1}{2}\sum_{i=1}^m (\xx - \mu_{\yy})(\xx-\mu_{\yy})^T.
  \]
  Hence,
  \[
  \Sigma = \frac{1}{m} \sum_{i=1}^m (\xx - \mu_{\yy})(\xx-\mu_{\yy})^T.
  \]
  
  We conclude that the maximum likelihood estimates of the parameters are given by
  \[
  \begin{aligned}
  	\phi &= \frac{1}{m} \sum_{i=1}^m 1\{\yy = 1\}, \\
  	\mu_0 &= \frac{\sum_{i=1}^{m} 1\{\yy = 0\} \xx }{\sum_{i=1}^m 1\{\yy = 0\}}, \\
  	\mu_1 &= \frac{\sum_{i=1}^{m} 1\{\yy = 1\} \xx }{\sum_{i=1}^m 1\{\yy = 1\}} ,\\ 
  	\Sigma &= \frac{1}{m} \sum_{i=1}^m (\xx - \mu_{\yy})(\xx - \mu_{iy})^T.
  \end{aligned}
  \]
    
  \item \textbf{Coding problem.}
  
  \item See jupyter notebook for plots.
  
  \item See jupyter notebook for plots. On Dataset 1 GDA perform worse than logistic regression. This might be the case because for Dataset 1, the distribution of features are not quite multivariate normal.
  
  \item *** TO-DO ***
    
\end{enumerate}
\end{solution}

\begin{problem}[Problem 2]
	\textbf{Incomplete, Positive-Only Labels}
	
	Dataset without full access to labels. In particular, we have labels only for a subset of positive examples. All negative examples and the rest of positive examples are unlabeled.
	
	Assume dataset $\{ (\xx, \tt, \yy) \}_{i=1}^m$ where $\tt \in \{0,1\}$ is true label and where 
	\[
	\yy = \begin{cases}
		1 & \xx \text{ is labeled}\\
		0 & \text{otherwise.}
	\end{cases}
	\]
	All labeled examples are positive, which is to say $p(\tt = 1 \mid \yy = 1) = 1$. Goal is to construct a binary classifier $h$ of true label $t$ which only access to partial labels $y$. That is, construct $h$ such that $h(\xx) \approx p(\tt = 1 \mid \xx)$ as closely as possible, using only $x$ and $y$.
\end{problem}

\begin{solution}
\begin{enumerate}[(a)]
	\item Suppose each $\yy$ and $\xx$ conditionally independent given $\tt$:
	\[
	p(\yy = 1 \mid \tt = 1, \xx) = p(\yy = 1 \mid \tt = 1).
	\]
	That is, labeled examples are selected uniformly at random from positive examples.
	
	Want to show $p(\tt = 1 \mid \xx) = p(\yy = 1 \mid \xx) / \alpha$ for some $\alpha \in \R$. As $p(\cdot \mid \xx)$ is a conditional measure, we have
	\[
	\begin{aligned}
		p(\yy = 1 \mid \xx) &= p(\yy = 1 \mid \tt = 1, \xx) p(\tt = 1 \mid \xx) \\
		& \qquad + p(\yy = 1 \mid \tt = 0, \xx) p(\tt = 0 \mid \xx) \\
		&= p(\yy = 1 \mid \tt = 1, \xx) p(\tt = 1 \mid \xx) \\
		&= p(\yy = 1 \mid \tt = 1) p(\tt = 1 \mid \xx).
	\end{aligned}
	\]
	Hence, $p(\tt = 1 \mid \xx) = p(\yy = 1 \mid \xx) / \alpha$ where $\alpha = p(\yy = 1 \mid \tt = 1)$.
	
	\item Estimate $\alpha$ using a trained classifier $h$ and a held-out validation set $V$. Let $V_+ = \{\xx \in V \mid \yy = 1\}$. Assuming $h(\xx) \approx p(\yy = 1 \mid \xx)$ for all $\xx$. Want to show
	\[
	h(\xx) \approx \alpha \text{ for all } \xx \in V_+.
	\]
	May assume that $p(\tt = 1 \mid \xx) \approx 1$ when $\xx \in V_+$.
	
	We have
	\[
	\begin{aligned}
		h(\xx) &\approx p(\yy = 1 \mid \xx) \\
		&= p(\yy = 1 \mid \tt = 1, \xx) p(\tt = 1 \mid \xx) \\
		&\approx \alpha.
	\end{aligned}
	\]
	
	\item \textbf{Coding problem.}
	
	\item \textbf{Coding problem.}
	
	\item \textbf{Coding problem.} Estimate the constant $\alpha$ using validation set.
  \[
  \alpha \approx \frac{1}{\abs{V_+}} \sum_{\xx \in V_+} h(\xx).
  \]
  To plot the decision boundary, we need to calculate the rescaled $\theta$, write $\theta_*$. The new decision boundary is given by $\frac{1}{\alpha} \frac{1}{1+ \exp(-\theta^T x)} = \frac{1}{2}$. We have
  \[
  \theta^T x + \log(\frac{2}{\alpha} - 1) = 0.
  \]
  This is equivalent to $\theta_*^T x = 0$. This shows that $\theta_*$ and $\theta$ differs only in the 0-th index by a constant $\log(\frac{2}{\alpha} - 1)$.
\end{enumerate}
\end{solution}

\begin{problem}[Problem 3]
  \textbf{Poisson Regression}
\end{problem}
\begin{solution}
\begin{enumerate}[(a)]
  \item The poisson distribution parametrized by $\lambda$ is 
  \[
  p(y ; \lambda) = \frac{e^{-\lambda} \lambda^y}{y!}.
  \]
  Therefore, we have
  \[
    p(y ; \lambda) = \frac{1}{y!} \exp(-\lambda + y \log \lambda).
  \]
  Compare with $p(y ; \eta) = b(y) \exp (\eta^T T(y) - a(\eta))$, we conclude that the poisson distribution is in the exponential family, with
  \[
  \begin{aligned}
    b(y) &= \frac{1}{y!}, \\
    T(y) &= y, \\
    \eta &= \log \lambda, \\
    a(\eta) &= e^\eta.
  \end{aligned}
  \]

  \item The canonical response function for the family
  \[
  \E[T(y); \eta] = \E[T(y); \eta] = \lambda = e^\eta.
  \]

  \item For a general linear model and a training set, the log likelihood
  \[
  \begin{aligned}
    \log p(\yy \mid \xx; \eta) &= \log b(y) \exp(\eta^T T(y) - a(\eta)) \\
    &= \log b(y) + \eta^T T(y) - a(\eta).
  \end{aligned}
  \]
  
  For our model with poisson responses y, we have
  \[
  \begin{aligned}
    \ell = \log p(\yy \mid \xx; \theta) &= - \log y! + (\theta^T \xx) \yy - \exp(\theta^T \xx).
  \end{aligned}
  \]

  Taking the derivative with respect to $\theta_j$, we have
  \[
  \pdv{\ell}{\theta_j} = \qty(\yy - \exp(\theta^T \xx)) \xx_j
  \]

  Hence, the stochastic gradient ascent update rule for learning using a GLM model with poisson response $y$ is 
  \[
  \begin{aligned}
    \theta_j &:= \theta_j + \alpha \pdv{\ell}{\theta_j} \\
    &:= \theta_j + \alpha \qty(\yy - \exp(\theta^T \xx)) \xx_j.
  \end{aligned}
  \]

  \item \textbf{Coding problem.} To predict the dataset, recall that the hypothese function for our model with poisson response $y$ is
  \[
  h_\theta(x) = \E[y \mid x] = e^{\eta} = e^{\theta^T x}.
  \]
  Also, for the model, we utilize batch gradient ascent:
  \[
  \theta_j := \theta_j + \frac{\alpha}{m} \sum_{i=1}^{m}  \qty(\yy - \exp(\theta^T \xx)) \xx_j.
  \]

\end{enumerate}
\end{solution}

\begin{problem}[Problem 4]
  \textbf{Convexity of Generalized Linear Models} 
  
  Investigate nice properties of GLM. Goal is to show that the negative log-likelihood (NLL) loss of a GLM is convex with respect to the model parameters.

  Recall that for exponential family distribution
  \[
  p(y ; \eta) = b(y) \exp(\eta^T T(y) - a(\eta)),
  \]
  where $\eta$ is the \emph{natural parameter} of distribution. Our approach is to show the Hessian of loss w.r.t the model parameters is PSD.

  Restrict to the case where $\eta$ is scalar and $\eta$ is modeled as $\theta^T x$. Assume $p(Y \mid X ; \theta) \sim \text{ExponentialFamily}(\eta)$ where $\eta \in \R$ is a scalar and $T(y) = y$. That is
  \[
  p(y ; \eta) = b(y) \exp(\eta y - a(\eta)).
  \]
\end{problem}

\begin{solution}
  \begin{enumerate}[(a)]
    \item The mean of the distribution
    \[
    \E[y; \eta] = \int y p(y ; \eta) dy = \int y b(y) \exp(\eta y - a(\eta)) dy.
    \]
    Following the hint, observe that
    \[
    \begin{aligned}
      \pd{\eta} \int p(y;\eta) dy &= \int \pd{\eta} p(y ; \eta) dy \\ 
      &= \int b(y) \qty(y - \pdv{a}{\eta}) \exp(\eta y - a(\eta)) dy.
    \end{aligned}
    \]
    While $\int p(y; \eta) dy = 1$, we have $\pd{\eta} \int p(y; \eta) = 0$ and 
    \[
    \E[y ; \eta] = \int b(y) \pdv{a(\eta)}{\eta} \exp(\eta y - a(\eta)) dy.
    \]
    Since $\pdv{a(\eta)}{\eta}$ does not depend on $y$, we have
    \[
    \E[y ;\eta] = \int b(y) \pdv{a(\eta)}{\eta} \exp(\eta y - a(\eta)) dy = \pdv{a(\eta)}{\eta} \int b(y) \exp(\eta y - a(\eta)) dy = \pdv{a(\eta)}{\eta}.
    \]
    This shows that $\E[Y \mid X; \theta]$ can be represented as the gradient of the log-partition function $a$ with respect to the natural parameter $\eta$.

    \item Notice that
    \[
    \begin{aligned}
      \pdv{\E[y; \eta]}{\eta} &= 
      \pd{\eta} \int y b(y) \exp(\eta y - a(\eta)) dy \\
      &= \int y b(y) \qty(y - \pdv{a}{\eta}) \exp(\eta y - a(\eta)) dy \\
      &= \int y^2 b(y) \exp(\eta y - a(\eta)) dy - \int y b(y) \pdv{a}{\eta} \exp(\eta y - a(\eta)) dy \\
      &= \E[y^2; \eta] - \pdv{a}{\eta} \E[y; \eta] \\
      &= \E[y^2; \eta] - (\E[y; \eta])^2 \\
      &= \Var(y; \eta).
    \end{aligned}
    \]
    This completes the proof, and we can see that $\Var(Y \mid X ; \theta)$ can be expressed as the second derivative of the mean w.r.t $\eta$ (i.e. the second derivative of log-partition function $a(\eta)$ w.r.t natural parameter $\eta$).

    \item The loss function $\ell(\theta)$, the NLL of the distribution
    \[
    \begin{aligned}
      \ell(\theta) &= - \log \prod_{i=1}^m p(\yy \mid \xx ; \eta) \\
      &= - \sum_{i=1}^{m} \log p(\yy \mid \xx ; \eta) \\
      &= \sum_{i=1}^{m} - \log b(\yy) - \eta \yy + a(\eta) \\
      &= \sum_{i=1}^{m} - \log b(\yy) - \yy \theta^T \xx + a(\theta^T \xx).
    \end{aligned}
    \]
    Now, to calculate the Hessian of the loss function w.r.t $\theta$, we first calculate
    \[
    \begin{aligned}
      \pdv{\ell}{\theta_k} = \sum_{i=1}^m \qty(\pdv{a}{\eta} - \yy) \xx_k.
    \end{aligned}
    \]
    It follows that
    \[
    \pdv{\ell}{\theta_j \theta_k} = \sum_{i=1}^m \pdv[2]{a}{\eta} \xx_j \xx_k.
    \]
    Hence, the Hessian of the loss function is 
    \[
    H = \sum_{i=1}^m \pdv[2]{a}{\eta} \xx (\xx)^T.
    \]

    To prove the Hessian is always PSD, consider any $z \in \R^n$, where $n$ is the dimension of $\xx$, and
    \[
    \begin{aligned}
      z^T H z &= \sum_{i=1}^m z_j H_{jk} z_k \\
      &= \sum_{i=1}^m \pdv[2]{a}{\eta} z_j x_j x_k z_k \\
      &= \sum_{i=1}^m \Var(Y \mid X; \eta) (x^T z)^2 \\
      &\geq 0,
    \end{aligned}
    \]
    since the variance is always non-negative. This completes the proof that NLL loss of GLM is convex.
  \end{enumerate}
\end{solution}

\begin{remark}
  \begin{itemize}
    \item Any GLM model is convex in its model parameters.
    \item The exponential family of probability distribution are mathematically nice. We can caucluate the means and variance using derivatives, which is easier that integrals.
  \end{itemize}
\end{remark}

\end{document}
